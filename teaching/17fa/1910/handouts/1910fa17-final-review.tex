\documentclass[]{handout}

\input{preamble}
\usepackage[margin=2.5cm]{geometry}

\begin{document}

\title{Final Exam Review}
\maketitle

\noindent\textit{\textbf{Warning:} These problems are by no means a comprehensive
	representation of the material that might appear on the exam.  That is, there may be topics not covered
	by these problems that you are still responsible for knowing.  Let these problems be a supplement to your
	preparation for the exam, but be sure to review other sources (e.g. your notes, homework assignments, and the
	textbook) as well.}


\begin{enumerate}[(1)]
	% ints : indefinite, definite, substitution, parts,
	%   trig sub, trig ints, improper ints, partial fracs

	\item Find the following  integrals.

	\begin{enumerate}

		\item $\displaystyle\int \frac{2- \cos x + \sin x}{\sin^2 x}\, dx$  % T 8.*.159

\begin{answer} 
		$\displaystyle -2 \cot x - \ln|\csc x + \cot x| + \csc x + C$
\end{answer}

		\

		\item $\displaystyle\int_{-\infty}^\infty \frac{2\, dx}{e^x + e^{-x}}$  % T 8.*.149

\begin{answer} $\pi$
\end{answer}

		\

		\item $\displaystyle\int \frac{dr}{1 + \sqrt r}$  % T 8.*.211

\begin{answer} $2 \sqrt r - 2\ln(1+ \sqrt r) + C$
\end{answer}

		\

		\item $\displaystyle\int_{\pi/4}^{\pi/2} \sqrt{1 + \cos 4x} \, dx$  % T 8.*.177

\begin{answer} $\displaystyle \frac{\sqrt 2}{2}$
\end{answer}

		\

		\item $\displaystyle\int x^3 \sin x \, dx$  % ~T 8.*.210

\begin{answer} $3(x^2 - 2)\sin x - x(x^2 - 6) \cos x + C$
\end{answer}

		\

		\item $\displaystyle\int (27)^{3\theta + 1}\, d\theta$  % T 8.*.209$

\begin{answer} 
		$\displaystyle \frac{1}{3} \left(\frac{27^{3 \theta + 1}}{\ln 27} \right) + C$
\end{answer}

		\

		\item $\displaystyle\int_{-1}^1 \frac{dy}{y^{2/3}}$  % T 8.*.137

\begin{answer} $6$
\end{answer}

		\

		\item $\displaystyle\int_2^\infty\frac{dx}{\sqrt{x}-\sqrt[4]{x}}$ % yy

\begin{answer} diverges
\end{answer}

		\

		\item $\displaystyle\int_2^4\frac{dx}{x\sqrt{x^2-4}}$ % yy

\begin{answer} $\pi/6$
\end{answer}

	\end{enumerate}


	% ders : inverse, inverse trig, hyp trig, FTC,

	\item Find the following derivatives.

	\begin{enumerate}

		\item $F'(x)$, where $\displaystyle F(x) = \int_{\tan^{-1} x}^{\pi/4} e^{\sqrt t}\, dt$
			% T 5.*.128

\begin{answer} 
		$\displaystyle - \frac{e^{\sqrt{\tan^{-1} x}}}{1+x^2}$
\end{answer}

		\

		\item $G'(x)$, where $\displaystyle G(x) = \int_{\ln x}^{\sinh^2 x} e^t \, dt$

\begin{answer} 
		$\displaystyle 2 e^{\sinh^2x}\sinh x \cosh x - 1$
\end{answer}

		\

		\item $(f^{-1})'(\pi/4 + 1)$, where $f(x) = x + \arctan x$.

\begin{answer} 
		$\displaystyle \frac 2 3$
\end{answer}

	\end{enumerate}

	\item	\begin{enumerate} % T 5.A.5
		\item Suppose that $\displaystyle \int_0^{x^2} f(t) \, dt = x \cos \pi x$.  Find $f(4)$.

\begin{answer} 
		$\displaystyle \frac 1 4$
\end{answer}

		\

		\item Suppose that $\displaystyle \int_0^{f(x)} t^2 \, dt = x \cos \pi x$.  Find $f(4)$.

\begin{answer} 
		$\displaystyle \sqrt[3]{12}$
\end{answer}

		\
	\end{enumerate}

	% area, volume (shells, washers), arclength, surface area

	\item Let $R$ be the ``triangular'' region in the first quadrant that is bounded above by
		the line $y = 1$, below by the curve $y = \ln x$, and on the left by $x = 1$.
		%  T 8.A.21,22
	\begin{enumerate}

		\item Find the area of the region $R$.

\begin{answer} 
		$e - 2$
\end{answer}

		\

		\item Find the volume of the solid obtained by rotating $R$ around the $x$-axis.

\begin{answer} 
		$\pi$
\end{answer}

		\

		\item Find the volume of the solid obtained by rotating $R$ around the line $x = 1$.

\begin{answer} 
		$\displaystyle\frac{\pi}{2}(5- 4e +e^2)$
\end{answer}

	\end{enumerate}

	\item The (infinite) region bounded by the coordinate axes and the curve $y = - \ln x$
		in the first quadrant is revolved about the $x$-axis to generate a solid.  Find
		the volume of the solid.  % T 8.A.24

\begin{answer}
		 $2 \pi$
\end{answer}

		\

	\item  \begin{enumerate}

		\item Find the length of the curve $y = \ln x$ from $x = 1$ to $x = e$. % T 8.A.27

\begin{answer} 
		$\displaystyle \sqrt{1+e^2} - \ln\left(\frac{\sqrt{1+e^2}}{e} + \frac{1}{e}\right) - \sqrt{2} + \ln(1+ \sqrt 2)$
\end{answer}

		\

		\item Find the surface area of the surface generated by rotating
			$y = \ln x$ from $x = 1$ to $x = e$ around the $y$-axis. % T 8.A.28

\begin{answer} 
		$\displaystyle \pi\left( - \sqrt 2 + e \sqrt{1+e^2} - \sinh^{-1}(1) + \sinh^{-1}(e) \right)$
\end{answer}

		\

	\end{enumerate}

	% work

	\item A reservoir shaped like a right circular cone, point down, 20 ft across on the top
		and 8 feet deep, is full of water.  How much work does it take to pump the
		water to a level of 6 feet above the top?  (The density of water is approximately
		62.4 lb/ft$^3$.)  % T 6.*.41

\begin{answer} 
		$\approx 418,208.81$ (ft-lb)
\end{answer}

		\

%	% diff eqs
%
%	\item Solve the following differential equations and initial value problems..
%
%	\begin{enumerate}
%
%		\item $\displaystyle \frac{dy}{dx} + 3x^2 y = x^2$, \quad $y(0) = -1$  % T 9.*.24
%
%\begin{answer} 
%		$\displaystyle y = \frac{1}{3}\left(1 - 4 e^{-x^3}\right)$
%\end{answer}
%
%		\
%
%		\item $\displaystyle (\sec x) \frac{dy}{dx} = e^{y+ \sin x}$ % T 9.1.18
%
%\begin{answer} 
%		$y = - \ln|C - e^{\sin x}|$
%\end{answer}
%
%		\
%
%		\item $\displaystyle \frac{dy}{dx} = \frac{y\ln y}{1 + x^2}$, \quad $y(0) = e^2$  % T 9.*.22
%
%\begin{answer} 
%		$\ln|y| = 2 e^{\arctan x}$.
%\end{answer}
%
%	\end{enumerate}

	% limits : lhopital, seqs

	\item Find the following limits.

	\begin{enumerate}

		\item $\displaystyle\lim_{n\to\infty} \left( 1 - \frac{1}{n} \right)^n$

\begin{answer} 
		$\displaystyle\frac{1}{e}$
\end{answer}

		\

		\item $\displaystyle\lim_{n\to\infty} \frac{1}{n} \int_1^n \frac{1}{x} \, dx$ % T 11.1.83

\begin{answer} 
		$0$
\end{answer}

		\

		\item $\displaystyle\lim_{n\to\infty} \sum_{k=1}^n \ln \sqrt[n]{1 + \frac k n}$.  % T 8.A.13

\begin{answer} 
		$\ln 4 - 1$
\end{answer}

%		\

%		\item $\displaystyle\lim_{n\to\infty} \frac{\ln(\sin(1/n))}{1 - \cos(1/n)}$ % custom


	\end{enumerate}

	% series

	\item Determine if the following series converge absolutely, converge conditionally,
		or diverge.

	\begin{enumerate}

		\item $\displaystyle \sum_{n=1}^\infty \frac{\ln n}{n^3}$

\begin{answer} 
		Converges absolutely.  (Direct comparison to $\sum (1/n^2)$.)
\end{answer}

		\

		\item $\displaystyle \sum_{n=3}^\infty \frac{\ln n}{\ln(\ln n)}$

\begin{answer} Diverges.  (Divergence test.)
\end{answer}

		\

		\item $\displaystyle \sum_{n=1}^\infty \frac{(-1)^n(n^2 + 1)}{2n^2 + n - 1}$

\begin{answer} 
		Diverges.  (Divergence test.)
\end{answer}

		\item $\displaystyle \sum_{n=1}^\infty \frac{(\arctan n)^2}{n^2 + 1}$

\begin{answer} 
		Converges.  (Limit comparison to $\sum(1/n^2)$.)
\end{answer}

		\

		\item $\displaystyle \sum_{n=2}^\infty \frac{\log_n(n!)}{n^3}$

\begin{answer} 
		Converges.  (Direct comparison to $\sum(1/n^2)$.)
\end{answer}

		\

		\item $\displaystyle\sum_{n=1}^\infty a_n$, where $a_1=2$, $a_{n+1}=\frac{6n+1}{5n+3}a_n$ % yy

\begin{answer} Diverges (Ratio Test.)
\end{answer}

		\

	\end{enumerate}

	% power / taylor series

	\item Find the interval of convergence of the following power series.

	\begin{enumerate}

		\item $\displaystyle \sum_{n=1}^\infty \frac{(-1)^{n-1}(3x-1)^n}{n^2}$  % T 11.*.43

\begin{answer} $[0,2/3]$
\end{answer}

		\

		\item $\displaystyle \sum_{n=1}^\infty (\mbox{csch}\, n) x^n$ % T 11.*.49

\begin{answer} $(-e,e)$
\end{answer}

		\

	\end{enumerate}

	\item Find the Maclaurin series for the given functions.

	\begin{enumerate}

		\item $f(x) = \cos(x^{5/2})$

\begin{answer} 
			$\displaystyle \sum_{n=0}^\infty \frac{x^{5n}}{(2n)!}$
\end{answer}

		\

		\item $\displaystyle f(x) = \frac{1}{(x-1)^2}$

\begin{answer} 
		$\displaystyle \sum_{n=1}^\infty n x^{n-1}$
\end{answer}

		\

	\end{enumerate}

	\item \begin{enumerate}

		\item Find the Maclaurin series of $\arctan x$.  What is the interval of
			convergence of the series?  (Hint: first find a
			series expansion for $(d/dx)(\arctan x)$.)

\begin{solution} We have
		\begin{align*}
			\frac{d}{dx}(\arctan x) = \frac{1}{1+x^2} = \sum_{n=0}^\infty (-1)^n x^{2n}
		\end{align*}
		for $|x| < 1$.  Since $\arctan 0 = 0$, it follows that
		\begin{align*}
			\arctan x = \int_0^x \frac{1}{1+t^2}\,dt
				= \sum_{n=0}^\infty \frac{(-1)^n x^{2n+1}}{2n+1}
		\end{align*}
		on the interval $(-1,1)$.
\end{solution}

		\

		\item Use the fact that $\tan(\pi/6) = 1/\sqrt{3}$ to express $\pi$ as the
			sum of an infinite series.

\begin{solution}
		 We have $\arctan(1/\sqrt 3) = \pi /6$.  Since $1/\sqrt 3$ is in the
		interval $(-1,1)$, we can use the series above.
		\begin{align*}
			\frac{\pi}{6} = \arctan(1/\sqrt{3}) =
				\sum_{n=0}^\infty \frac{(-1)^n}{3^{(2n+1)/2}(2n+1)}
		\end{align*}
		Hence
		\begin{align*}
			\pi = \sum_{n=0}^\infty \frac{6 \cdot (-1)^n}{3^{(2n+1)/2}(2n+1)}.
		\end{align*}
\end{solution}

	\end{enumerate}


	\item Let $f(x) = \sqrt{1+x}$. % yy
	\begin{enumerate}
		\item Find $f''(x)$ and $f''(0)$.

\begin{answer}
		 $f''(x) = -\frac{1}{4}(1+x)^{-3/2}$, and $f''(0) = -1/4$.
\end{answer}

		\

		\item Find the first three terms of the Maclaurin series of $f(x)$.

\begin{answer} 
		$1 + \frac{1}{2} x - \frac{1}{8} x^2$
\end{answer}

		\

		\item Find the first three \emph{nonzero} terms of the Maclaurin series of $\int_0^x\sqrt{1+t^3}\,dt$.

\begin{answer}
		 $x + \frac{1}{8} x^4 - \frac{1}{56} x^7$
\end{answer}

	\end{enumerate}



\end{enumerate}

\end{document}
